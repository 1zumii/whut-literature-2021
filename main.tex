\documentclass[fontset=ubuntu,zihao=-4,a4paper]{ctexart}

%% 目录
\usepackage{titletoc}
\titlecontents{section}[3.8em]
{\songti\zihao{-4}}
{\contentslabel{4em}}{\hspace*{-4em}}
{~\titlerule*[0.8pc]{$.$}~\contentspage}

%% 字体
\setmainfont{Times New Roman}
%\setmonofont{Source Code Pro}
%\setmonofont{Droid Sans Mono}
\setmonofont[Path=font/]{Monaco.ttf}
\newcommand{\zhongsong}{\CJKfontspec[Path=font/]{STZhongSong.ttf}}	%华文中宋
\ctexset{
	section = {
        % format = \centering\bfseries\zihao{-2} \heiti,
		format = \centering\zihao{-2} \heiti,
		name = {第, 章}
	},
	subsection = {
        % format = \bfseries\zihao{3} \heiti
		format = \zihao{3} \heiti
	},
	subsubsection = {
        % format = \bfseries\zihao{4} \heiti
		format = \zihao{4} \heiti
	}
}

%% 交叉引用
\usepackage{hyperref}

%% 代码块
\usepackage{listings}
\usepackage{color}
\definecolor{lightgray}{RGB}{244,247,248}
\definecolor{darkgray}{rgb}{.4,.4,.4}
\definecolor{purple}{rgb}{0.65, 0.12, 0.82}

\lstdefinelanguage{JavaScript}{
  keywords={typeof, new, true, false, catch, function, return, null, catch, switch, var, if, in, while, do, else, case, break},
  keywordstyle=\color{purple}\bfseries,
  ndkeywords={class, export, boolean, throw, implements, import, this},
  ndkeywordstyle=\color{red}\bfseries,
  identifierstyle=\color{black},
  sensitive=false,
  comment=[l]{//},
  morecomment=[s]{/*}{*/},
  commentstyle=\color{darkgray}\ttfamily,
  stringstyle=\color{blue}\ttfamily,
  morestring=[b]',
  morestring=[b]"
}

\lstset{
   language=JavaScript,
   backgroundcolor=\color{lightgray},
   extendedchars=true,
   basicstyle=\footnotesize\ttfamily,
   showstringspaces=false,
   showspaces=false,
   numbers=left,
   numberstyle=\footnotesize,
   numbersep=9pt,
   tabsize=2,
   breaklines=true,
   showtabs=false,
   captionpos=b
}

%% 图形支持宏包
\usepackage{graphicx}               % 嵌入png图像,修改表格的整体大小
\usepackage{booktabs}               % 三线表格中的上中下直线线型设置宏包

%% 数学公式
% 学位论文版权使用授权书那边的方块可能用到
\usepackage{amsmath}                % AMS LaTeX宏包
\usepackage{amssymb}                % 用来排版漂亮的数学公式
\usepackage{amsbsy}
\usepackage{mathrsfs}               % 英文花体字体

%% 页边距调整
\usepackage{setspace}				%设置间距
\usepackage{calc}                   %包含textwidth,textheight
\usepackage{geometry}                
\geometry{top=2.5cm,bottom=2cm,left=2.5cm,right=2cm}

%% 设置间距
\setlength{\lineskip}{20pt}                     %设置行间距
\setlength{\parskip}{0.5\baselineskip - 10pt}   %设置段间距

%% 页眉页脚样式的定义方式
\usepackage{fancyhdr}
\pagestyle{fancy}
\fancyhf{}  
\fancyhead[C]{\zihao{5}  \kaishu 武汉理工大学毕业设计(论文)}
\fancyfoot[C]{~\zihao{5} \thepage~}
\renewcommand{\headrulewidth}{0.65pt} 

%%%%%%%%% 正文区 %%%%%%%%%
\begin{document}
%% 论文引用格式
\bibliographystyle{gbt7714-2005}     
%% 封面和前言
%% 封面
\smallskip
\vspace*{1.7cm}
\begin{center}
\begin{figure}[!th]
\centering
\includegraphics[width=0.7\linewidth]{figure/SchoolName}
\end{figure}

\vspace*{1.0cm}
\zhongsong{\zihao{1} 毕业设计(论文)} \\
\vspace*{4.0cm}
 \heiti{\zihao{2} 基于区块链的考试系统的设计与开发}\\
\vspace*{4.0cm}
\zhongsong
% \songti
\begin{tabular}{cc}
 \zihao{-2} 学院(系):&\underline{\makebox[7cm][c]{\zihao{-2}计算机科学与技术学院}} \\ 
 \\
 \zihao{-2}专业班级: & \underline{\makebox[7cm][c]{\zihao{-2}软件zy1701}} \\ 
 \\
 \zihao{-2}学生姓名: & \underline{\makebox[7cm][c]{\zihao{-2}林雨钦}} \\ 
 \\
 \zihao{-2}指导教师: & \underline{\makebox[7cm][c]{\zihao{-2}向广利}} \\ 
 \\
\end{tabular} 
\end{center}
\thispagestyle{empty}
\clearpage

%%%%%%%%% 原创性声明 %%%%%%%%%
\begin{center}
\zihao{-2} \textbf{学位论文原创性声明}
\end{center}

本人郑重声明:所呈交的论文是本人在导师的指导下独立进行研究所取得的研究成果。除了文中特别加以标注引用的内容外,本论文不包括任何其他个人或集体已经发表或撰写的成果作品。本人完全意识到本声明的法律后果由本人承担。 
\begin{flushright}
\zihao{4} 作者签名:\qquad ~~~\\

年\qquad 月\qquad 日
\end{flushright}
\vskip 2cm
\begin{center}
\zihao{-2} \textbf{学位论文版权使用授权书}
\end{center}

本学位论文作者完全了解学校有关保障、使用学位论文的规定,同意学校保留并向有关学位论文管理部门或机构送交论文的复印件和电子版,允许论文被查阅和借阅。本人授权省级优秀学士论文评选机构将本学位论文的全部或部分内容编入有关数据进行检索,可以采用影印、缩印或扫描等复制手段保存和汇编本学位论文。\smallskip

本学位论文属于
\begin{tabular}[t]{l}
1、保密$ \Box$,在~~~年解密后适用本授权书  \\ 
2、不保密$ \Box$  \\ 
\end{tabular} \\
\begin{center}
(请在以上相应方框内打“$\surd”$)
\end{center}
\begin{flushright}
\zihao{4} 作者签名:  \quad\quad\quad\quad 年 \quad  月  \quad  日\\
导师签名:   \quad\quad\quad\quad 年 \quad  月 \quad   日\\
\end{flushright}
\thispagestyle{empty}
\clearpage
\pagestyle{plain}
\pagenumbering{Roman}
\section*{摘要}
在利益证明(PoS)和许可区块链中,由验证者组成的委员会同意并签署每个新的交易区块。这些区块由网络中的所有用户进行验证、传播和存储。然而,变节攻击(Posterior Corruption)对这些设计构成了常见的威胁,因为对手可以在委员会验证者认证了一个区块之后破坏他们,并使用他们的签名密钥来认证另一个区块。设计高效、安全的数字签名用于PoS区块链,可以大幅降低节点的带宽、存储和计算需求,从而实现更高效的应用。

我们提出了一个基于配对的前向安全多签名方案Pixel,该方案经过优化,可用于区块链中,实现了带宽、存储需求和验证工作的大幅节省。Pixel 签名由两个组元组成,无论签名者数量多少,都可以使用三个配对和一个指数进行验证,并支持将单个签名非交互式聚合成一个多签名。Pixel签名也是前向安全的,并且让签名者随着时间的推移演化他们的密钥,这样新的密钥就不能用于在旧的区块上签名,保护了区块链的变节攻击击。我们展示了如何将Pixel集成到任何PoS区块链中。接下来,我们在真实世界的PoS区块链实现中评估了Pixel,表明它在存储、带宽和区块验证时间方面产生了显著的节省。特别是,Pixel签名将1500笔交易的区块大小减少了35\%,区块验证时间减少了38\%。
\addcontentsline{toc}{section}{摘要}    % 目录中加入

\clearpage  % 分页

\section*{Abstract}
In Proof-of-Stake (PoS) and permissioned block chains, a committee of verifiers agrees and sign every new block of transactions. These blocks are validated, propagated, and stored by all users in the network. However, posterior corruptions pose a common threat to these designs, because the adversary can corrupt committee verifiers after they certified a block and use their signing keys to certify a different block. Designing efficient and secure digital signatures for use in PoS blockchains can substantially reduce bandwidth, storage and computing requirements from nodes, thereby enabling more efficient applications.We present Pixel, a pairing-based forward-secure multi-signature  scheme  optimized  for  use  in  blockchains, that achieves substantial savings in bandwidth, storage requirements, and verification effort.Pixel signatures consist of two group elements, regardless of the number of signers, can be verified using three pairings and one exponentiation, and support non-interactive aggregation of individual signatures into a multi-signature.Pixel signatures are also forward-secure and let signers evolve their keys over time, such that new keys cannot be used to sign on old blocks, protecting against posterior corruptions attacks on blockchains. We show how to integrate Pixel into any PoS blockchain. Next, we evaluate Pixel in a real world PoS blockchain implementation, showing that it yields notable savings in storage, bandwidth, and block verification time. In particular,Pixel signatures reduce the size of blocks with1500transactions by35\%and reduce block verification time by 38\%.

\addcontentsline{toc}{section}{Abstract}    % 目录中加入
%% 目录
\pagestyle{empty}
\tableofcontents 
\thispagestyle{empty}
\newpage                            %使得后面页码从1开始

%% 论文正文
\pagestyle{fancy}

\pagenumbering{arabic}                          %后面用阿拉伯数字记页码
\setlength{\lineskip}{20pt}                     %设置行间距
\setlength{\parskip}{0.5\baselineskip - 10pt}   %设置段间距

\section{绪论}
\subsection{背景研究}
2016年7月,中共中央、国务院办公厅印发了《国家信息化发展战略纲要》,规范和指导了未来10年中央对国家信息化的指导思想和战略方针。信息技术的基础水平和相关产业的发展程度,与国家信息化进程密切相关。《国信纲要》提倡,要加强信息技术领域前沿和基础研究,同时应用相应技术,打造协同发展的产业生态。2020年5月两会期间,多位人大代表和政协委员提出了,关于区块链技术在产业中的应用、促进产业发展升级和数字货币监督监管问题,三大方面的想法与提议。由此可知,区块链已然成为了当下乃至今后十年内,我国科技、信息领域发展的“顶层设计”。在这样的发展潮流下,区块链的落地应用和技术推广如雨后春笋:国家邮政局鼓励以区块链为主要应用核心的新一代技术升级;银保监会为提高数据安全性,在交易环节应用区块链技术;教育部也在《全国大中小学教材建设规划(2019-2022年)》的记者问答中表示,下一步将围绕区块链等新技术领域重点建设,集中力量编写、打造新经典教材。

2020年初的疫情导致近乎所有学生都参与到了线上教学,由此影响,在线课堂、在线考试受到了极大的推广。腾讯会议、钉钉、腾讯课堂等线上视频会议的产品快速迭代之中逐渐成熟。与之相应,也诞生了许多在线考试的相关解决方案,例如超星学习通、中国大学慕课、智慧树。线上考试的方式与传统的线下纸质化考试相比,有如下的优劣之处:

\begin{itemize}
    \item 能够代替教师的部分工作,一定程度上地方便学校、教育机构对考试的安排管理,提高了工作效率;
    \item 有效地缩短考试前期安排的组织周期,减轻了相关通知传递过程中的人力损耗;
    \item 线上交卷、评卷缩短了学生得到考试反馈结果的时间周期;
    \item 在相关硬件设施条件满足的前提下,进一步地保证了考试过程中的公平公正;
    \item 存在线上系统都无法避免的网络攻击风险;
    \item 数据集中化存储管理,有潜在的被篡改等安全问题;
\end{itemize}

如上的在线考试解决方案通常使用了移动端 C/S 结构,桌面端 B/S 结构。如此的客户端-服务端解构模式,实现了在不同的用户前端实现下,用户数据的共通。在这样的项目架构模型中,数据的安全性、可靠性、可信度便显得尤为关键。结合这两个方面,提出本项目「基于区块链技术的线上考试系统」。线上考试系统的方式合理利用了当今时代互联网的便捷性与普及程度。采用国家大力支持推广的区块链技术,可以有效地保证数据存储管理的安全性。

区块链技术最早由中本聪于《比特币:一种点对点的电子现金系统》\upcite{ref0}一文中提出,是一种类似于分布式的账本的数据存储与管理技术。在共识机制、工作量证明(PoW)等方式保障下,区块链技术具有:不可篡改、可追溯源、去中心化等特性与优点。这些特点正好与线上考试系统的数据安全方面的问题十分契合:防篡改、可溯源保证了考试数据的安全;去中心化的特性发挥了数据分布式存储的优势,降低了集中化数据存储方式存在的安全风险,减轻了系统服务器压力,加强了数据可用性、系统可用性。将区块链技术应用至系统中,会是诸如线上考试系统此类的数据安全敏感系统今后的更新趋势,也呼应了国家对于区块链技术重视与号召。


\subsection{需求分析}
\subsubsection{目标用户群体}
\begin{itemize}
    \item 学校、教育机构:有固定时间周期性的测试或考试需要,使用线上考试系统可以覆盖一些简单的定期多次小测验;
    \item 学生:无纸化的线上考试系统使得考试结果反馈更加快速;
\end{itemize}

\subsubsection{用户主要目的}
需要一款线上考试系统的解决方案,能方便快捷的满足学校、教育机构平时周期性的测试的需要。教师能够简单的操作就能配置题目、题库、发布试卷,通过预先设定的参考答案,能够自动评卷登分,将考试结果与相关题目解析快速反馈给参与考试的学生。

\subsubsection{功能性需求}
\begin{figure}[htb!]
    \centering
    \includegraphics[width=0.8\linewidth]{_images/功能性需求.png}
    \caption{功能性需求}
    \label{功能性需求}
\end{figure}
\paragraph{用户角色分类} 系统的参与者主要分为三类:学生、教师、管理员,学生与教师对应实际业务场景中的角色,管理员是因系统的操作与非业务的操作而存在。
\paragraph{题库功能} 题目作为考卷的基本单位,考卷就是多个题目的组成。项目需要题目的基本管理功能。题目按题型又将分为:单选题、多选题、判断题、填空题等。题目按照学科分类,因此又需要题库功能对不同的学科题目进行管理。
\paragraph{考卷功能} 考卷是考试进行的必要组成部分,考卷需要由教师手动从题库中选择,或是通过设定一定的规则从设定好的题库中抽取自动组卷。考卷也需要基本的增删改查功能。
\paragraph{考试功能} 考试通过选择已创建的考卷而开启,学生可以进入已创建的考试进行线上考试。考试也需要设置起始时间和结束时间属性。
\paragraph{评卷功能} 题目设置的时候提供对应的正确答案选项,使得在作答结束后能够通过正确答案与学生的作答情况自动评卷计分,并提供预先设置的参考题解给学生。
\paragraph{用户管理} 管理员才被允许对系统中的用户进行新增、删除、修改信息、查看。用户也可以对自己的基本信息做修改,例如修改密码、头像、昵称。
\paragraph{权限分类} 根据以上的用户角色分类和功能模块,需要将具体的功能模块访问限制与用户角色对应。学生只允许参与考试和查看自己考试的情况;老师具有学生的所有权限用于考卷的试测,此外需要考卷功能模块、题库功能模块、评卷功能模块的权限;管理员具有所有的功能模块权限,即老师的权限加上用户管理功能模块的权限。

\subsubsection{非功能性需求}
\begin{itemize}
    \item 界面美观,布局简洁大方;
    \item 用户交互操作友好。用户意外的错误操作,或是发生运行时错误,应有合理的信息提示;
    \item 系统使用需要有清晰、完善的文档说明,项目代码注释完善、易读;
    \item 项目需要具有足够的测试性,以保证系统性能可用于实际环境;
    \item 项目结构设计良好、易于维护,并留存有一定后期功能拓展的空间;
    \item 用户操作具有完备的鉴权流程,防止非法操作;
    \item 前端布局弹性,根据不同的分辨率响应式渲染展示;
\end{itemize}


\subsection{进度安排}
\paragraph{2021/1/8 - 2021/2/28} 确定选题,查阅文献,外文翻译和撰写开题报告。这期间主要通过导师提供的相关材料,以及自己查阅相关的文献,熟悉项目题目的研究背景、应用领域、技术生态。在完成了文献查阅和外文文献翻译后,根据自己的这阶段对项目选题的方方面面的理解,撰写开题报告;
\paragraph{2021/3/1 - 2021/3/20} 完成系统核心功能设计,主要包括数据库表设计,对项目中实体的抽象;使用 Vue 及其相关技术栈实现最小 Demo,主要目的是通过实际开发实践巩固所学的前端技术部分的理论知识;熟悉 SpringBoot、Fisco Bcos 等后端部分的技术文档,对业务对象、功能模块的设计实现有一定思考和初步设想,能做到胸有成竹;
\paragraph{2021/3/21 - 2021/4/20} 参考网上成熟的系统界面原型设计,对前端部分的布局和组件抽象有大概的构想;编写项目的核心功能模块的代码,包括前端页面展示部分和后端数据处理部分;
\paragraph{2021/4/21 - 2021/5/11} 尝试与实践区块链的结点部署,并修改项目后端核心实体对象的数据处理,增加区块链接入层,实现安全敏感的重要数据上链;
\paragraph{2021/5/12 - 2021/5/18} 熟悉相关的测试技术,对项目进行单元测试、压力测试等各方面的测试,并记录数据;
\paragraph{2021/5/19 - 2021/5/31} 撰写及修改毕业论文;
\paragraph{2021/6/1 - 2021/6/5} 准备答辩。

\subsection{论文结构安排}
\paragraph{第一章}绪论介绍了在当下的社会政策与技术潮流中,项目的研究和应用意义。以及分析了项目面向的用户群体,用户需求。

\paragraph{第二章}简要介绍了项目所使用的相关技术栈,简要概述了技术选型的思考过程:选用的技术的稳定程度与社区生态;与相似技术及其技术栈对比之中的优劣;与项目开发的适用性。

\paragraph{第三章}介绍了系统的整体设计模型、分层的构思过程,项目系统具体的分工模块,模块间的相互协作。

\paragraph{第四章}重点着眼于项目系统的几个核心模块的代码实现思路,和模块的代码开发中所遇到的难点、个人的思考与解决方式。

\paragraph{第五章}展示了使用相关的测试技术对项目应用时各方面的测试结果与结论。

\paragraph{第六章}总述个人对于项目仍存在的不足之处,和潜在的改进空间。
\section{相关技术}
\subsection{Fisco Bcos}
区块链按照类型可划分为:公有链、私有链、联盟链。公有链对全网公开,默认公开化,任何人都能成为其中的节点,访问其中的数据。就如同国家高速,任何人都能随意的开车驶入。而私有链的使用和部署范围则与公有链相反,交易、验证操作都被局限于一个企业的规模之内,具有一定的封闭性、良好的隐私性、数据的安全性。由于私有链的规模特点和共识机制特点,交易的手续费(gas)与公有链相比较低,交易速度(或称为出块速度)亦是公有链无法相比的。联盟链则可视为公有链和私有链的结合,同时包括了两者的特点。联盟链在本质上更接近于私有链,其应用范围为更大范围的组织之内,可能由众多机构组成,但到底仍是由十分有限的结点组成,因此联盟链具有的是部分去中心化的特点。此外,联盟链可控性较强。由于节点数的规模缘故,交易速度也介于公有链、私有链之间,但也远快于公有链。如上特点与项目系统对于敏感数据上链的需求十分匹配。联盟链在实际项目中节点成员都存在于同一的受信任组织内,根据实际的项目系统需要修改区块链上规则,部署新的合约等操作也更为方便。故而,系统的后端区块链部分选用联盟链的技术是十分合适的。

现今海外区块链技术的开源设计框架,数 IBM 的 Fabric 较为成熟,也有活跃的社区生态。而自2018年3月始,中美贸易争端不断,更有甚者,对我国多项核心技术施压。因此未雨绸缪,支持国内区块链技术的发展,选用 Fisco Bcos 作为项目的区块链技术框架。Fisco Bcos 是由金融区块链合作联盟(金链盟)开源的一款区块链底层框架,是受工信部认可的三大开源区块链底层平台之一。具有深度定制的安全性、可控性的特点,根据实际需要,实现符合国内需要的独特设计。社区生态丰富,应用领域广泛,截至2020年5月,共有上千家的企业与机构、逾万名社区成员参与共建共治,发展成为最大最活跃的国产开源联盟链生态圈。

\subsection{SpringBoot}
Spring 是目前后端领域生态最为活跃,且发展迅猛的开源框架。Spring 是为了解决企业级应用开发的复杂性而诞生。Spring 为简化 Java 的开发,采用了以下几种关键策略:
\begin{itemize}
    \item 基于 POJO(Plain Ordinary Java Object)的轻量级,实现最小侵入性编程;
    \item 通过控制反转(Invert of Control)和依赖注入(Dependency Injection)等特性实现接口编程的低耦合;
    \item 面向切面编程(AOP);
\end{itemize}
然而随着 Spring 的技术生态发展和社区活跃度不断上升,Spring 被应用于越来越多的实际生产项目中,涉及领域愈发广泛。因而,开发一个 Spring 项目需要整合的文件也随之变得复杂、繁琐,与其设计初衷背道而驰。

脱胎于繁琐配置的 Spring,SpringBoot 是由 Pivota 团队提供的全新框架,其设计目的旨在简化 Spring 应用的初始化构建以及开发过程。SpringBoot 的理念是“约定大于配置”,且有如下的特点:
\begin{itemize}
    \item 开箱即用,从根本上实现更快的入门体验;
    \item 内嵌 Tomcat、Jetty 或 Undertow,使得开发者无需额外部署 WAR;
    \item 提供了丰富的封装好的 starter 依赖,用于简化开发过程中的构建配置;
    \item 自动配置 Spring,且具有良好的扩展性。如有需要,简单配置便可引入第三方库;
    \item 无代码生成、无需编写XML;
\end{itemize}

SpringBoot 具有丰富的项目应用范例、良好的社区生态与开发体验,因此选用 SpringBoot 作为项目后端的技术框架,能够通过现有的优秀项目,参考学习其设计、构想;活跃的社区也能在实际的项目开发过程中避雷避坑,给一些疑难问题提供解决思路。

\subsection{SpringData JPA}
历史车轮滚滚向前,最初 Java 后端开发框架技术栈的三驾马车 SSH(Struts、Spring、Hibernate)这一 JavaEE 框架风靡一时。而随着技术更新迭代,其中的 Struts 被 SpringMVC 所替代,再后来随着开箱即用、自动配置的 SpringBoot 的出现逐渐让开发者从 Spring + SpringMVC 的阵营中转投。Hibernate 作为技术栈中的数据持久化 ORM 框架,也出现了许多优秀的横向竞品,例如 MyBatis、JPA。这些 ORM 框架各有优劣,而 MyBatis 更受国内大部分的开发者欢迎。然而,MyBatis 的无论是 XML 写法,亦或是注解的使用方式,mapper 的代码编写总是让人觉得略有心烦。相较之下,SpringData JPA 无需 mapper 的使用方式更为简洁。

SpringData JPA 作为 Spring 官方推荐的数据库访问组件,简化了各种数据库的访问操作。在 SpringData JPA 中数据访问层对象(DAO)的实现,是通过 Repository 作为 DAO 层接口来操作实体类所对应的数据表。Repository 模式是领域驱动设计中另一个经典的模式。当一个接口继承 JpaRepository 接口之后便自动具备了一系列常用的数据操作方法。因此,在抽象出业务流程中的实体(Entity)以后,在其实体类加上  \verb!@Entity! 的注解,便可以将实体类与 Repository 接口相关联,并提供了基础的增删改查,乃至分页等,实际业务中常见的基础数据库操作。这样的实现极大的省去了后端开发中的一部分基础代码编写,减轻了后端开发过程中的心智负担。
\section{系统设计}
\subsection{总体架构模型设计}
\subsubsection{前端架构模型}
\begin{figure}[htb]
    \centering
    \includegraphics[width=\linewidth]{_images/前端模型.png}
    \caption{前端架构模型}
\end{figure}
Vue 的技术基础是将根节点组件挂载在页面的一个 DOM 元素上,而根节点组件由可由多个组件组成,如此向下细分成子组件。组件系统是 Vue 的另一个重要概念,因为它是一种抽象,允许我们使用小型、独立和通常可复用的组件构建大型应用。

组件的数据来源可以划分为:Props、Data、Computed。父子组件之间的数据通信通过 Props,而 Vue 的设计理念中,数据是自上而下单向流动。

面对需要多组件之间共享公共的数据的场景,需要引入 Vuex 的 store。store 将托管的公有数据 state,通过预先在根节点的注册,注入到需要的组件的 Props 中。如有需要对 store 中的数据进行修改,可以将 store 的 mutation 注入到组件的 Methods 中,通过提交(commit)mutation 实现对 store 中的 state 修改。

组件的 Data 也可能来自于用户的交互产生,又或是向后端服务请求的数据。组件中的请求方法,通过调用工具函数 Requester 向后端服务发起请求,其中如有必要应进行 Token 用户令牌的检查。响应得到的数据将返回给请求方法,进而给组件的 Data 赋予新值。

组件的渲染可能受控于父组件的逻辑,也可能受路由器的控制。在父组件中注册成为 <router-view> 的子组件,就通过 Vue-Router 的路由器,根据页面的 URL 动态响应渲染。其中也可以通过页面的 URL 动态路由匹配,向组件的 Props 注入匹配到的参数。


\subsubsection{后端架构模型}
\begin{figure}[htb]
    \centering
    \includegraphics[width=0.85\linewidth]{_images/后端模型.png}
    \caption{后端架构模型}
\end{figure}
前端请求进入后端服务首先会被拦截器 Interceptor 所截获。通过 InterceptorConfig 配置需要拦截的 api 的 url 规则,并加入对应的 Interceptor。项目中的 JWT 鉴权流程便放在拦截器这一层运行。

请求根据具体 api 的 URL,进入不同的 controller,controller 根据业务调用对应的 Service 中的方法。由于使用了 SpringData JPA,数据库中的表与 Repository 对应且关联,因而 Service 中对 DAO 的操作则需要依赖于 Repository。

针对敏感数据需要上链的 Service,通过 Fisco Bcos 提供的 SDK,接入预先编译成 Java 文件的智能合约。而智能合约由通过密钥对的形式连接至区块链节点组成的网络。

\subsection{模块划分}
\begin{figure}[htb]
    \centering
    \includegraphics[width=\linewidth]{_images/功能模块划分.png}
    \caption{功能模块划分}
\end{figure}
项目系统根据功能划分成几个功能模块:题库模块、用户模块、考卷模块、考试模块。

按照权限分类用户角色,其对应的功能模块为:
\begin{itemize}
    \item 学生:考试模块
    \item 老师:题库模块、考试模块、考卷模块
    \item 管理员:题库模块、考试模块、考卷模块、用户模块
\end{itemize}

\subsection{Http 请求响应}
\subsubsection{前端请求}
\noindent\textit{front/Requester.js/post函数}
\begin{lstlisting}
function post(url, params = {}, needToken = true) {
    const {token} = TokenManager.getToken()
    ... ...
    Logger.log('request', {url, params})
    return axios({
        method: 'post',
        url: url,
        responseType: 'json',
        data: params,
        headers
    }).then(response => {
        const {data, status, statusText} = response
        Logger.log('response', data)
        return data
    }).catch(reason => {
        const {status, statusText} = reason.response
        Logger.error('response', status, statusText, reason.response)
        throw reason.response
    })
}
\end{lstlisting}
前端的请求发起主要依赖于基于 xhr 实现的第三方库 axios。axios 由于请求的异步延时特性,所以使用了 Promise 进行了封装。

前端对后端的请求过程中,请求的 url 前缀部分都是相同的。此外,对于常用的 POST、GET 方法,请求头部分也有相似的部分,例如,POST 请求统一使用「application/json」的 content-type,并将具体的请求数据 json 序列化后,通过 axios 的 data 字段写入请求体中。因而,可以对 axios 再做一层定制,封装成项目的工具函数集合 Requester 中 post、get。

在业务流程中,使用 Requester,仅仅需要根据请求的方法,调用对应的 post、get,传入对应的 api 的 url、请求参数 params,和是否需要 token 验证的标志位 needToken。而无需关心其中具体的请求配置,以及响应的错误处理。获取响应以后再通过一层 Promise 解构其中真正需要的响应数据,使得在业务层面对于请求-响应中的细节是透明无感知的,从实际意义上的减轻了业务开发过程中的心智负担。并可以在工具函数 Requester 中加入日志输出,以便在线上部署后,通过日志迅速定位错误信息。

\subsubsection{后端响应}
\noindent\textit{back/ResultVO.java}
\begin{lstlisting}[language=Java] 
@Data
@JsonInclude(JsonInclude.Include.NON_NULL) 
public class ResultVO<T> {

    public ResultVO(Integer code, String msg, T data) {
        this.code = code;
        this.msg = msg;
        this.data = data;
    }

    public ResultVO() {}

    private Integer code;

    private String msg = "";

    private T data;
}
\end{lstlisting}

\noindent\textit{back/ExamController.java/getExamRecordList函数}
\begin{lstlisting}[language=Java]
@GetMapping("/record/list")
@ApiOperation("获取当前用户的考试记录")
ResultVO<List<ExamRecordVo>> getExamRecordList(HttpServletRequest request) {
    ResultVO<List<ExamRecordVo>> resultVO;
    try {
        // 拦截器里设置上的用户id
        String userId = (String) request.getAttribute("user_id");
        // 下面根据用户账号拿到他(她所有的考试信息),注意要用VO封装下
        List<ExamRecordVo> examRecordVoList = examService.getExamRecordList(userId);
        resultVO = new ResultVO<>(0, "获取考试记录成功", examRecordVoList);
    } catch (Exception e) {
        e.printStackTrace();
        resultVO = new ResultVO<>(-1, "获取考试记录失败", null);
    }
    return resultVO;
}
\end{lstlisting}

后端的响应部分更多的是借助 SpringBoot 中 \lstinline!starter-web! 启动器所提供的相关框架能力。针对响应体的自定义封装,主要是使用了 ResultVO 这一值对象(Value Object)。其中定义了针对业务而言的状态码 code,即业务操作成功为 0,业务操作失败则为非 0。随之附带 msg 作为扩展的信息说明字段。并且 \lstinline!ResultVO! 通过泛型 \lstinline!<T>! 注入具体响应时的数据类型。例如,\lstinline!getExamRecordList! 函数中展示的,针对当前业务操作需要的是 \lstinline!List<ExamRecordVo>! 的数据类型,因而,在 try-catch 前初始化的响应结果 \lstinline!resultVO! 就是 \lstinline!ResultVO<List<ExamRecordVo>>!。

\subsection{鉴权设计}
针对前后端分离的项目结构,鉴权设计主要是借助 JWT(JavaScript Web Token)这一通用的解决方案完成。
\subsubsection{前端鉴权}
\noindent\textit{front/TokenManager.js}
\begin{lstlisting}[language=JavaScript]
// 设置 Token
function setToken({token, userInfo}, remember = false) {
    if (remember) {
        // 写入 localStorage
        localStorage.setItem(...)
    }
    // 在 store 中设置值
    store.commit(...)
    ... ...
}

// 获取 Token
function getToken() {
    let token, userInfo
    if (store.getters.getToken) {
        token = store.getters.getToken
    }
    if (store.state.auth.userInfo) {
        userInfo = store.state.auth.userInfo
    }
    if (!token && localStorage.getItem(tokenKey)) {
        token = localStorage.getItem(tokenKey)
        userInfo = JSON.parse(localStorage.getItem(userInfoKey))
        // 在 store 中设置值
        store.commit(...)
        ... ...
    }
    return {
        token, userInfo
    }
}

// 移除 Token
function removeToken() {
    ... ...
}
\end{lstlisting}
前端部分对于 Token,以及可以与 Token 视作相关联的敏感用户信息 UserInfo,都使用相同的存储管理思路。

针对登录时勾选了“记住登录状态”的情况,将 Token 以及 UserInfo 写入浏览器提供的 localStorage 中。localStorage 是浏览器提供的一种缓存能力,写入 localStorage 中的键值对,可以持久化存储,使得在浏览器退出后也不会丢失。

针对普通的登录情况,则将这些重要数据写入 Vuex 提供的 store 中,方便各组件获取和修改。

如果重新打开浏览器进入页面,且有勾选“记住登录状态”的情况,store 中已有的 Token 是因为进程退出而丢失的,然而 localStorage 中可能还存有未过期的 Token。因而,在 TokenManager 的 getToken 函数中,需要首先检查 store 中是否有保存 Token,如果没有则再在 localStorage 中读取。如果 localStorage 中读取成功,则说明用户是有勾选了“记住登录状态”,则需要将 Token 作为函数返回值之前,将读取到的 Token 存储到 store 中。

Token 存储的情况有多种,但是移除的时候无需关心是否存在,只需要一并清空 store 和 localStorage 中可能存在的键值对即可。

\noindent\textit{front/Requester.js/post函数}
\begin{lstlisting}[language=JavaScript]
function post(url, params = {}, needToken = true) {
    const {token} = TokenManager.getToken()
    if (needToken && !token) {
        Logger.error('not found token')
        return Promise.reject({
            code: 1401,  // 浏览器的401
            msg: 'not found token'
        })
    }
    const headers = {}
    if (needToken) {
        headers['Access-Token'] = `bearer ${token}`
    }
    ... ...
}
\end{lstlisting}

在 Requester 工具函数中,请求如果设置了 needToken 的标志位,则使用 TokenManager 的 getToken 函数中读取可能存在的 Token。如此设计,在请求的代码编写中,将两个模块解耦,仅仅通过函数调用相互关联,符合“高内聚,低耦合”的设计原则。

如果需要 Token 而 getToken 无法返回有效 Token 时,则需要进行错误的处理。遵照 axios 的 Promise 风格 api,错误处理也使用相似的 Promise.reject。其中自定义状态码设置为“1401”,意图借 HTTP 状态码的 401 相同的含义,再前加上 1,以示区别。

getToken 成功返回 Token 后,则通过请求头中的 Access-Token 字段,在请求中携带上 Token。

\noindent\textit{front/Requester.js/handleRequestError函数}
\begin{lstlisting}[language=JavaScript]
function handleRequestError(error) {
    if (
        (
            error && typeof (error) === 'object' &&
            error.code > 1400 && error.code < 1500
        ) || error.status === 401
    ) {
        TokenManager.removeToken()
    }
    Logger.error(error)
}
\end{lstlisting}

对于如上的需要 Token,而又没有 Token 的特殊处理情况,则通过 handleRequestError 对于前面定义的特殊状态码“1401”做处理动作,在当前的项目中,仅仅只是调用了 removeToken。但是,将这个针对 Token 的错误处理环节的独立抽离,也是意图方便以后可能会加入的新的错误处理逻辑,使得项目代码留存有一定的扩展空间。

\subsubsection{后端鉴权}
\noindent\textit{back/LoginInterceptor.java/preHandle函数}
\begin{lstlisting}[language=Java]
@Override
public boolean preHandle(... ...) throws Exception {
    ... ...
    // 注意要和前端适配Access-Token属性,前端会在登陆后的每个接口请求头加Access-Token属性
    String token = request.getHeader("Access-Token");
    ... ...
    if (token != null) {
        // 请求中是携带参数的
        Claims claims = JwtUtils.checkJWT(token);
        if (claims == null) {
            // 返回null说明用户篡改了token,导致校验失败
            sendJsonMessage(response, JsonData.buildError("token无效,请重新登录"));
            return false;
        }
        ... ...
        return true;
    }
    ... ...
    return false;
}
\end{lstlisting}
后端的鉴权与前端部分相对应,而后端的请求鉴权主要是通过 SpringBoot 提供的拦截器 \lstinline!Interceptor! 框架能力完成。拦截器通过配置对应的拦截规则,调用对应的拦截器,可以实现需要鉴权的请求在进入实际的业务代码 \lstinline!Controller! 层之前,在拦截器中进行鉴权。将其这部分鉴权逻辑单独通过拦截器实现,目的是为了同业务代码独立开,互相不影响。如此的设计,也是遵从了“低耦合”的原则思想。
\begin{lstlisting}[language=Java]
public class JwtUtils {
    // 构建 token 的主题
    private static final String SUBJECT = ... ...;
    // 过期时间为1天
    private static final long EXPIRE = 1000 * 60 * 60 * 24;

    private static final String APP_SECRET = ... ...;

    public static String genJsonWebToken(User user) {
        ... ...
        return Jwts.builder().setSubject(SUBJECT)
                // 下面3行设置 token 中间字段,携带用户的信息
                ... ...
                // 设置过期时间
                ... ...
                // 生成的结果字符串太长,这里压缩下
                .compact();
    }

    /* 校验 token */
    public static Claims checkJWT(String token) {
        ... ...
    }
}
\end{lstlisting}

将 jsonwebtoken 提供的 api 能力针对项目的业务情况再进行一次封装,成为 \lstinline!JwtUtils! 工具。对外仅提供了对 Token 的创建、校验能力。

\subsection{数据库结构设计}
\subsubsection{数据库概念设计}
\begin{figure}[htb]
    \centering
    \includegraphics[width=\linewidth]{_images/ER图.png}
    \caption{数据库ER图}
\end{figure}

\subsubsection{数据库逻辑设计}

\section{系统实现}
\subsection{题库模块}
\begin{figure}[hb]
    \centering
    \includegraphics[width=\linewidth]{_images/题库模块截图.jpeg}
    \caption{题库模块}
\end{figure}
进入题库模块前需要先检查是否有进入的权限,如果没有则需要一定的 403 反馈。正常操作流程下,没有权限是无法通过点击事件进入题库,但也可能通过浏览器的地址栏 URL,通过路由器强行渲染出对应的题库 \lstinline!View! 组件,因此仍是需要在进入模块前进行权限检查。

如果有权限进入,则展示所有的题目,并可以进行分页操作、选择题库查看、选择具体题目操作。

分页操作,则需要在组件中托管具体的 \lstinline!current!、\lstinline!pageSize!、\lstinline!total! 数据,并注入 ant-d-v 提供的 \lstinline!<Pagination>!。分页请求时,向后端发起 \lstinline!/exam/question/list! 的 \lstinline!GET! 请求,并将 \lstinline!current! 和 \lstinline!pageSize! 作为参数。在后端 Controller 中接收请求,做处理。

\begin{lstlisting}[language=Java]
@GetMapping("/question/list")
@ApiOperation("获取问题的列表")
ResultVO<QuestionPageVo> getQuestionList(
    @RequestParam("pageNo") Integer pageNo, 
    @RequestParam("pageSize") Integer pageSize
) {
    ResultVO<QuestionPageVo> resultVO;
    
    try {
        QuestionPageVo questionPageVo = examService.getQuestionList(pageNo, pageSize);
        resultVO = new ResultVO<>(0, "获取问题列表成功", questionPageVo);
    } catch (Exception e) {
        e.printStackTrace();
        resultVO = new ResultVO<>(-1, "获取问题列表失败", null);
    }
    
    return resultVO;
}
\end{lstlisting}

选择具体题库,则展示具体题库下的题目。也可以进行修改题库信息、删除题库或新增题库。\upcite{ref12,ref14}

修改题库信息,可以修改题库的名称,或新增、删除其中包含的题目。

删除题库的操作,会将所有当前题库下的题目,与之解除关联。在具体的题库-题目 mapper 表中删除对应的映射。而不会删除具体的题目,即使已经没有任何题库与这些题目关联。但是这些题目仍是会在题库添加的时候,可选展示待加入。

新增题库的操作则会要求填写题库的基本信息,而后可以选择需要添加入题库的题目。也可以在创建成功以后,再通过修改题库的功能,对题库的题目做管理。

对于具体题目的操作,可以查看题目的详情,也可以对题目配置的属性做修改调整。

\begin{lstlisting}[language=Java]
@PostMapping("/question/update")
@ApiOperation("更新问题")
ResultVO<String> questionUpdate(@RequestBody QuestionVo questionVo) {
    // 完成问题的更新
    System.out.println(questionVo);
    try {
        examService.updateQuestion(questionVo);
        return new ResultVO<>(0, "更新问题成功", null);
    } catch (Exception e) {
        e.printStackTrace();
        return new ResultVO<>(-1, "更新问题失败", null);
    }
}
\end{lstlisting}

\begin{figure}[htb]
    \centering
    \includegraphics[width=\linewidth]{_images/题库模块.png}
    \caption{题库模块}
\end{figure}


\subsection{考卷模块}
进入考卷模块前需要先检查是否有进入的权限,如果没有则需要一定的 403 反馈。正常操作流程下,没有权限是无法通过点击事件进入考卷模块,但也可能通过浏览器的地址栏 URL,通过路由器强行渲染出对应的题库 \lstinline!View! 组件,因此仍是需要在进入考卷模块前进行权限检查。

\begin{lstlisting}[language=JavaScript]
export default {
  ... ...
  computed: {
      ... ...
      storeUserInfo: function(){
          return TokenManager.getToken().userInfo
      },
      ... ...
  },
  methods: {
      queryExamPaper: function(pageIndex){
          ... ...
          const userId = this.storeUserInfo?.id
          ... ...
          Requester.post(...)
          ... ...
      }
  },
  ... ...
}
\end{lstlisting}
如果有权限进入考卷模块,则需要从 store 中获取当前用户的 \lstinline!id!,并传入后端,并携带分页所需要的 \lstinline!current!、\lstinline!pageSize! 等信息。

在页面内可以进行创建考卷的操作。首先需要填写一些考卷的基本信息,例如考卷名称等。然后,需要去选择是否从题库中自动生成考卷的题目。如果选择从题库中自动生成,需要选择对应的题库,并且配置需要多少的客观题,及其中具体的题型和分值。如果需要手动选择添加的题目,则会展示系统中已有的题目,供手动选择。配置好考卷的基础信息以及具体的题目或选择的题库后,则向后端发起创建考卷的请求。生成考卷成功后,返回展示当前用户创建的考卷列表。

修改考卷的操作,可以修改考卷的基本信息,也可通过勾选框批量删除考题。或是通过弹窗查看未添加的考题,批量勾选添加。

删除考卷的操作,仅仅只删除考卷在数据库表中的记录,与考卷相关联的题目并不会被删除,即便是已经不归属与任何的考卷。
\begin{figure}[htb]
    \centering
    \includegraphics[width=\linewidth]{_images/考卷模块.png}
    \caption{考卷模块}
\end{figure}


\subsection{用户管理模块}
渲染用户模块前需要先检查是否有进入的权限,如果没有则需要一定的 403 反馈。正常操作流程下,没有权限是无法通过点击事件进入用户模块,但也可能通过浏览器的地址栏 URL,通过路由器强行渲染出对应的题库 View 组件,因此仍是需要在进入用户模块前进行权限检查。

进入用户模块后,以列表形式展示所有的用户。并可以通过 Select 框选择,按照不同的用户角色进行条件筛选,展示对应用户的情况。并且支持分页能力。

新增用户时需要填写用户的基本信息,并且需要选择用户的角色,赋予其权限划分。用户是通过 username 来区分的,所以在注册用户时,需要先判断 username 是否已存在,如果存在,则返回“用户名已存在”。如果不存在,将新用户信息添加到数据库。

修改用户和新增用户类似,也需要判断新的 username 是否已存在,如果不存在,则将修改后的信息更新到数据库。

删除用户仅仅将用户从 user 表中删除,而不删除所有相关联的其余表中的 creatorId(或将其设置为 -1)。这样就能保证其他原先创建的考卷、题库、题目,不会受到其创建人的用户被删除,而出现异常情况。其 creatorId 的失效也仅仅影响了查询部分。\upcite{ref7,ref22,ref17}

\begin{figure}[htb]
    \centering
    \includegraphics[width=0.7\linewidth]{_images/用户管理模块.png}
    \caption{用户管理模块}
\end{figure}


\subsection{考试模块}
所有的用户都有权限进入考试模块。首先展示的是所有的考试,所以在进入页面前会先请求所有的考试信息。
\begin{lstlisting}[language=JavaScript]
function getExamCardList() {
  return axios({
    url: api.ExamCardList,
    method: 'get',
    headers: {
      'Content-Type': 'application/json;charset=UTF-8'
    }
  })
}
\end{lstlisting}
\begin{lstlisting}[language=Java]
@GetMapping("/card/list")
@ApiOperation("获取考试列表,适配前端卡片列表")
ResultVO<List<ExamCardVo>> getExamCardList() {
    // 获取考试列表卡片
    ResultVO<List<ExamCardVo>> resultVO;
    try {
        List<ExamCardVo> examCardVoList = examService.getExamCardList();
        resultVO = new ResultVO<>(0, "获取考试列表卡片成功", examCardVoList);
    } catch (Exception e) {
        e.printStackTrace();
        resultVO = new ResultVO<>(-1, "获取考试列表卡片失败", null);
    }
    return resultVO;
}
\end{lstlisting}
其中,可以进行条件筛选。可以按照已结束的考试、参加过的考试、可参加的仍在有效时间内的、未开始的,进行对考试列表的筛选。如果是展示的参加过的考试,点击事件会展示考试的详情。包括考试的答案、问题的正确答案、正确答案的解析,以及考试的得分。

在考试列表中,可以选择考试参加。点击选择的考试将跳转到答题系统中,即可进行答题。答题过程中将会进行计时操作。提交答案前,将会进行检查是否有未作答的题目,如有则拒绝提交。提交考试后将跳转回考试模块的最初页面。

\begin{figure}[htb]
    \centering
    \includegraphics[width=\linewidth]{_images/答题系统.png}
    \caption{答题系统}
\end{figure}

创建考试并不是所有用户角色都拥有的权限。因而,创建考试的按钮需要根据当前登录的用户角色,通过 
\lstinline!v-if! 的方式决定是否渲染。点击创建考试,需要选择考试使用的考卷,以及填写考试的基本信息、起始时间、结束时间,以及时间限制。\upcite{ref18}

\begin{figure}[htb]
    \centering
    \includegraphics[width=\linewidth]{_images/考试模块.png}
    \caption{考试模块}
\end{figure}



\section{系统测试}
\subsection{用户管理模块测试用例}
\begin{table}[h!]
\begin{tabularx}{\linewidth}{@{}Xlc@{}}
\toprule
测试用例        & 预期结果        & 是否达到预期 \\ \midrule
新增用户        & 新增账户成功      & 是      \\
新增用户:输入已存在的用户名 & 提示已存在       & 是      \\
登录:输入错误密码    & 登录失败,提示密码错误 & 是      \\
登录:输入不存在的用户名 & 提示用户不存在     & 是      \\
修改用户信息:输入已存在的用户名 & 提示已存在 & 是 \\
修改个人信息      & 修改成功        & 是      \\ \bottomrule
\end{tabularx}
\end{table}


\subsection{考试模块测试用例}
\begin{table}[h!]
\begin{tabularx}{\linewidth}{@{}Xlc@{}}
\toprule
测试用例           & 预期结果      & 是否达到预期 \\ \midrule
参加考试:不答题直接交卷   & 提示未完成所有题目 & 是      \\
参加考试:未答完所有题目交卷 & 提示未完成所有题目 & 是      \\
参加考试:完成所有题目交卷  & 交卷成功      & 是      \\
查询成绩:选择参加过的考试  & 显示考试详情    & 是      \\
参加考试:点击考试参加    & 进入答题系统    & 是   \\ \bottomrule
\end{tabularx}
\end{table}

\subsection{题库模块测试用例}
\begin{table}[h!]
\begin{tabularx}{\linewidth}{@{}Xlc@{}}
\toprule
测试用例           & 预期结果      & 是否达到预期 \\ \midrule
新增题目:不填写题目信息 & 提示必填项 & 是 \\
新增题目:不选择题型 & 提示必填项 & 是  \\
新增题目:填写信息,选择题型 & 新增成功 & 是 \\
查看题目:所有 & 展示所有题目 & 是 \\
查看题目:搜索 & 展示符合搜索内容的题目 & 是 \\
修改题目:修改信息 & 修改成功 & 是 \\
修改题目:未填写题目信息 & 提示必填项 & 是 \\
修改题目:不填写题目说明 & 修改成功 & 是  \\
新增题库:不填写任何内容   & 提示必填项 & 是      \\
新增题库:不勾选题目 & 新增成功 & 是      \\
新增题库:填写对应内容,并勾选题目    & 新增题库成功    & 是   \\ 
修改题库:清空所有信息  & 提示必填项      & 是      \\
修改题库:清空勾选题目  & 修改题库成功    & 是      \\
删除题库 & 删除题库成功 & 是  \\ \bottomrule
\end{tabularx}
\end{table}

\subsection{考卷模块测试用例}
\begin{table}[h!]
\begin{tabularx}{\linewidth}{@{}Xlc@{}}
\toprule
测试用例           & 预期结果      & 是否达到预期 \\ \midrule
新增考卷:不填写任何内容   & 提示必填项 & 是      \\
新增考卷:不填写名称 & 提示必填项 & 是      \\
修改考卷:不修改考卷信息  & 没有发生修改      & 是      \\
删除考卷  & 删除成功    & 是      \\
新增考卷:不勾选对应题目 & 提示组卷 & 是  \\
新增考卷:填写对应内容    & 新增考卷成功    & 是   \\ \bottomrule
\end{tabularx}
\end{table}


\subsection{章节总结}
通过如上的测试用例和数据结构,可以得出结论,基于区块链技术的考试系统,其功能与性能方面均达到用户需求,与最初需求设计要求目标一致。
\section{总结与展望}
\subsection{总结}
中美的贸易冲突带动了国内对于技术壁垒的警惕,大力发展自己的核心技术,才能避免在现如今最重要的科技、制造、技术领域受人制约。支持使用国产的区块链底层开发框架 Fisco Bcos 也是出于这样的目的。使用前后端分离的架构模式,给项目系统保证了一定的扩展性和功能性。基于区块链技术实现线上考试系统,充分发挥了区块链技术在数据安全方面的显著能力。借助互联网的便捷高效,操作简单,使得学校的教学工作不受到时间、空间的限制,极大地提高了学校、老师教学管理的工作效率,是未来的趋势。
\subsection{展望}
由于项目开发时间的限制,与自身技术水平有限,当前项目的进度也仅仅只是完成了项目系统的最基础的功能范围,仍存有很多改进的空间。例如可以在用户管理模块中,增加对科目的管理。考卷模块中,自动组卷的算法也可以进一步的深入研究。鉴权的部分,为增加更为流畅的用户体验,可以增加过期时限和刷新 Token 时限。对于区块链技术发展潮流的展望,现如今区块链技术的应用领域涉猎无数,技术生态蓬勃发展。我相信,在国家和社区、企业的大力支持下,我国的区块链技术一定会有领先全球的一天!

% \addcontentsline{toc}{section}{参考文献}
% \bibliography{bibfile}
% \clearpage

\section*{致谢}
\addcontentsline{toc}{section}{致谢}
毕业设计是我大学四年的最终篇章,不仅是我四年来的学习与努力的成果,也记录着同学、老师们对我学业上的帮助与关心。

我要感谢我的毕设导师,是向老师在项目初期的开展与探索试验阶段,给予我鼓励和指导,敦促我项目的进度情况。此外,我也要感谢苏学长和杜学长,在我项目开发过程中,遇上的技术难点,是他们的耐心讲解和循循善诱的启发,让我攻克难关。其中的讨论与思考,也让我受益匪浅。

大学四年是我人生中最为重要的篇章之一,是我最值得怀念的回忆。感谢我的父母,感谢我的同学们,感谢他们在我大学四年对我的支持与陪伴!

\end{document}
