\section{系统测试}
\subsection{用户管理模块测试用例}
\begin{table}[h!]
\begin{tabularx}{\linewidth}{@{}Xlc@{}}
\toprule
测试用例        & 预期结果        & 是否达到预期 \\ \midrule
新增用户        & 新增账户成功      & 是      \\
新增用户:输入已存在的用户名 & 提示已存在       & 是      \\
登录:输入错误密码    & 登录失败,提示密码错误 & 是      \\
登录:输入不存在的用户名 & 提示用户不存在     & 是      \\
修改用户信息:输入已存在的用户名 & 提示已存在 & 是 \\
修改个人信息      & 修改成功        & 是      \\ \bottomrule
\end{tabularx}
\end{table}


\subsection{考试模块测试用例}
\begin{table}[h!]
\begin{tabularx}{\linewidth}{@{}Xlc@{}}
\toprule
测试用例           & 预期结果      & 是否达到预期 \\ \midrule
参加考试:不答题直接交卷   & 提示未完成所有题目 & 是      \\
参加考试:未答完所有题目交卷 & 提示未完成所有题目 & 是      \\
参加考试:完成所有题目交卷  & 交卷成功      & 是      \\
查询成绩:选择参加过的考试  & 显示考试详情    & 是      \\
参加考试:点击考试参加    & 进入答题系统    & 是   \\ \bottomrule
\end{tabularx}
\end{table}

\subsection{题库模块测试用例}
\begin{table}[h!]
\begin{tabularx}{\linewidth}{@{}Xlc@{}}
\toprule
测试用例           & 预期结果      & 是否达到预期 \\ \midrule
新增题目:不填写题目信息 & 提示必填项 & 是 \\
新增题目:不选择题型 & 提示必填项 & 是  \\
新增题目:填写信息,选择题型 & 新增成功 & 是 \\
查看题目:所有 & 展示所有题目 & 是 \\
查看题目:搜索 & 展示符合搜索内容的题目 & 是 \\
修改题目:修改信息 & 修改成功 & 是 \\
修改题目:未填写题目信息 & 提示必填项 & 是 \\
修改题目:不填写题目说明 & 修改成功 & 是  \\
新增题库:不填写任何内容   & 提示必填项 & 是      \\
新增题库:不勾选题目 & 新增成功 & 是      \\
新增题库:填写对应内容,并勾选题目    & 新增题库成功    & 是   \\ 
修改题库:清空所有信息  & 提示必填项      & 是      \\
修改题库:清空勾选题目  & 修改题库成功    & 是      \\
删除题库 & 删除题库成功 & 是  \\ \bottomrule
\end{tabularx}
\end{table}

\subsection{考卷模块测试用例}
\begin{table}[h!]
\begin{tabularx}{\linewidth}{@{}Xlc@{}}
\toprule
测试用例           & 预期结果      & 是否达到预期 \\ \midrule
新增考卷:不填写任何内容   & 提示必填项 & 是      \\
新增考卷:不填写名称 & 提示必填项 & 是      \\
修改考卷:不修改考卷信息  & 没有发生修改      & 是      \\
删除考卷  & 删除成功    & 是      \\
新增考卷:不勾选对应题目 & 提示组卷 & 是  \\
新增考卷:填写对应内容    & 新增考卷成功    & 是   \\ \bottomrule
\end{tabularx}
\end{table}


\subsection{章节总结}
通过如上的测试用例和数据结构,可以得出结论,基于区块链技术的考试系统,其功能与性能方面均达到用户需求,与最初需求设计要求目标一致。