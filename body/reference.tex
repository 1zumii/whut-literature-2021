\section*{参考文献}
\addcontentsline{toc}{section}{参考文献}

\begingroup
% 隐藏自带的title
\def\section*#1{}

\begin{thebibliography}{99}

\songti\zihao{5}

\bibitem{ref0}Nakamoto S . Bitcoin: A Peer-to-Peer Electronic Cash System[J]. consulted.
\bibitem{ref1}杨骏佶. 面向计算机硬件的远程虚拟实验服务[D]. 江苏大学, 2019.
\bibitem{refAdd2}李建忠, 胡新刚, 孟志强,等. 运用虚拟化,云计算搭建集团企业云[J]. 创新世界周刊, 2019(Z1):78-85.
\bibitem{ref3}杨胜超~张瑞军. 基于二分图最优匹配算法的毕业论文选题系统[D]. 计算机系统应用, 2008.
\bibitem{ref4}杨晓. 云师大课程考试系统的设计与实现[D]. 电子科技大学, 2013.
\bibitem{refAdd5}杜剑楠, 胡德葳. 公共部门数据公开与再利用的政策指引——以欧盟PSI指令为例[J]. 西北工业大学学报·社会科学版, 2017(37):66-69.
\bibitem{ref6}孙阳. 南师大泰州学院教学评估管理信息系统的设计与实现[D]. 电子科技大学, 2014.
\bibitem{ref7}高淼. 电路板故障诊断系统TPS运行及数据管理模块设计与实现[D]. 电子科技大学, 2011.
\bibitem{refAdd71}Cachin C. Architecture of the hyperledger blockchain fabric[C]//Workshop on distributed cryptocurrencies and consensus ledgers. 2016, 310(4).
\bibitem{refAdd8}林洁璇. 信息技术网络考试系统的研究与设计[J]. 电脑知识与技术, 2013(09):2089-2091.
\bibitem{refAdd9}班珂, 黄丹. 两种Java Web通用开发框架的比较研究[J]. 电脑知识与技术, 2010, 06(019):5249-5251.
\bibitem{refAdd10}李辉忠, 李陈希, 李昊轩,等. FISCO BCOS技术应用实践[J]. 信息通信技术与政策, 2020(1).
\bibitem{refAdd11}Mark Heckler. Spring Boot: Up and Running[M]. " O'Reilly Media, Inc.". Sebastopol, 2021
\bibitem{ref12}姚迪. 河南电力公司网络培训考试系统的设计与应用[D]. 华北电力大学, 2017.
\bibitem{ref13}肖敏, 郭秋萍, 莫祖英. 政府数据开放发展历程及平台建设的差异分析——基于四个国家的调[D]. 图书馆理论与实践, 2019.
\bibitem{ref14}徐英. 基于java的数学自测评估系统的设计与实现[D]. 电子科技大学, 2013.
\bibitem{ref15}Drijvers M, Gorbunov S, Neven G, et al. Pixel: Multi-signatures for consensus[C]//29th {USENIX} Security Symposium ({USENIX} Security 20). 2020: 2093-2110.
\bibitem{refAdd16}孙铁翔, 朱基钗.中共中央办公厅、国务院办公厅印发《国家信息化发展战略纲要》[J]. 电子政务, 2016, 000(017):56-56.
\bibitem{ref17}夏婷婷. 基于FPGA的嵌入式环境监测系统设计[D]. 西北大学, 2015.
\bibitem{ref18}周高嵚. 基于白箱测试的源代码在线评测系统[D]. 北京化工大学, 2005.
\bibitem{refAdd19}Filipova O. Learning Vue. js 2[M]. Packt Publishing Ltd, 2016.
\bibitem{refAdd20}傅莞龙, 张传武, 彭安金. 使用Spring Data和JPA在JavaEE系统中简化持久层[J]. 电子世界, 2017(6).
\bibitem{refAdd21}Pollack M, Gierke O, Risberg T, et al. Spring Data: modern data access for enterprise Java[M]. " O'Reilly Media, Inc.", 2012.
\bibitem{ref22}佘春华. 基于认知灵活性理论的高中物理虚拟实验教学平台的设计与开发[D]. 广西师范学院, 2011.
\bibitem{ref23}Wang R, Ye K, Meng T, et al. Performance Evaluation on Blockchain Systems: A Case Study on Ethereum, Fabric, Sawtooth and Fisco-Bcos[C]//International Conference on Services Computing. Springer, Cham, 2020: 120-134.

\end{thebibliography}

\endgroup
\clearpage
