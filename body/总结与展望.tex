\section{总结与展望}
\subsection{总结}
中美的贸易冲突带动了国内对于技术壁垒的警惕,大力发展自己的核心技术,才能避免在现如今最重要的科技、制造、技术领域受人制约。支持使用国产的区块链底层开发框架 Fisco Bcos 也是出于这样的目的。使用前后端分离的架构模式,给项目系统保证了一定的扩展性和功能性。基于区块链技术实现线上考试系统,充分发挥了区块链技术在数据安全方面的显著能力。借助互联网的便捷高效,操作简单,使得学校的教学工作不受到时间、空间的限制,极大地提高了学校、老师教学管理的工作效率,是未来的趋势。
\subsection{展望}
由于项目开发时间的限制,与自身技术水平有限,当前项目的进度也仅仅只是完成了项目系统的最基础的功能范围,仍存有很多改进的空间。例如可以在用户管理模块中,增加对科目的管理。考卷模块中,自动组卷的算法也可以进一步的深入研究。鉴权的部分,为增加更为流畅的用户体验,可以增加过期时限和刷新 Token 时限。对于区块链技术发展潮流的展望,现如今区块链技术的应用领域涉猎无数,技术生态蓬勃发展。我相信,在国家和社区、企业的大力支持下,我国的区块链技术一定会有领先全球的一天!