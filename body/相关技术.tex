\section{相关技术}
\subsection{Fisco Bcos}
区块链按照类型可划分为:公有链、私有链、联盟链。公有链对全网公开,默认公开化,任何人都能成为其中的节点,访问其中的数据。就如同国家高速,任何人都能随意的开车驶入。而私有链的使用和部署范围则与公有链相反,交易、验证操作都被局限于一个企业的规模之内,具有一定的封闭性、良好的隐私性、数据的安全性。由于私有链的规模特点和共识机制特点,交易的手续费(gas)与公有链相比较低,交易速度(或称为出块速度)亦是公有链无法相比的。联盟链则可视为公有链和私有链的结合,同时包括了两者的特点。联盟链在本质上更接近于私有链,其应用范围为更大范围的组织之内,可能由众多机构组成,但到底仍是由十分有限的结点组成,因此联盟链具有的是部分去中心化的特点。此外,联盟链可控性较强。由于节点数的规模缘故,交易速度也介于公有链、私有链之间,但也远快于公有链。如上特点与项目系统对于敏感数据上链的需求十分匹配。联盟链在实际项目中节点成员都存在于同一的受信任组织内,根据实际的项目系统需要修改区块链上规则,部署新的合约等操作也更为方便。故而,系统的后端区块链部分选用联盟链的技术是十分合适的。

现今海外区块链技术的开源设计框架,数 IBM 的 Fabric 较为成熟,也有活跃的社区生态。而自2018年3月始,中美贸易争端不断,更有甚者,对我国多项核心技术施压。因此未雨绸缪,支持国内区块链技术的发展,选用 Fisco Bcos 作为项目的区块链技术框架。Fisco Bcos 是由金融区块链合作联盟(金链盟)开源的一款区块链底层框架,是受工信部认可的三大开源区块链底层平台之一。具有深度定制的安全性、可控性的特点,根据实际需要,实现符合国内需要的独特设计。社区生态丰富,应用领域广泛,截至2020年5月,共有上千家的企业与机构、逾万名社区成员参与共建共治,发展成为最大最活跃的国产开源联盟链生态圈。

\subsection{SpringBoot}
Spring 是目前后端领域生态最为活跃,且发展迅猛的开源框架。Spring 是为了解决企业级应用开发的复杂性而诞生。Spring 为简化 Java 的开发,采用了以下几种关键策略:
\begin{itemize}
    \item 基于 POJO(Plain Ordinary Java Object)的轻量级,实现最小侵入性编程;
    \item 通过控制反转(Invert of Control)和依赖注入(Dependency Injection)等特性实现接口编程的低耦合;
    \item 面向切面编程(AOP);
\end{itemize}
然而随着 Spring 的技术生态发展和社区活跃度不断上升,Spring 被应用于越来越多的实际生产项目中,涉及领域愈发广泛。因而,开发一个 Spring 项目需要整合的文件也随之变得复杂、繁琐,与其设计初衷背道而驰。

脱胎于繁琐配置的 Spring,SpringBoot 是由 Pivota 团队提供的全新框架,其设计目的旨在简化 Spring 应用的初始化构建以及开发过程。SpringBoot 的理念是“约定大于配置”,且有如下的特点:
\begin{itemize}
    \item 开箱即用,从根本上实现更快的入门体验;
    \item 内嵌 Tomcat、Jetty 或 Undertow,使得开发者无需额外部署 WAR;
    \item 提供了丰富的封装好的 starter 依赖,用于简化开发过程中的构建配置;
    \item 自动配置 Spring,且具有良好的扩展性。如有需要,简单配置便可引入第三方库;
    \item 无代码生成、无需编写XML;
\end{itemize}

SpringBoot 具有丰富的项目应用范例、良好的社区生态与开发体验,因此选用 SpringBoot 作为项目后端的技术框架,能够通过现有的优秀项目,参考学习其设计、构想;活跃的社区也能在实际的项目开发过程中避雷避坑,给一些疑难问题提供解决思路。

\subsection{SpringData JPA}
历史车轮滚滚向前,最初 Java 后端开发框架技术栈的三驾马车 SSH(Struts、Spring、Hibernate)这一 JavaEE 框架风靡一时。而随着技术更新迭代,其中的 Struts 被 SpringMVC 所替代,再后来随着开箱即用、自动配置的 SpringBoot 的出现逐渐让开发者从 Spring + SpringMVC 的阵营中转投。Hibernate 作为技术栈中的数据持久化 ORM 框架,也出现了许多优秀的横向竞品,例如 MyBatis、JPA。这些 ORM 框架各有优劣,而 MyBatis 更受国内大部分的开发者欢迎。然而,MyBatis 的无论是 XML 写法,亦或是注解的使用方式,mapper 的代码编写总是让人觉得略有心烦。相较之下,SpringData JPA 无需 mapper 的使用方式更为简洁。

SpringData JPA 作为 Spring 官方推荐的数据库访问组件,简化了各种数据库的访问操作。在 SpringData JPA 中数据访问层对象(DAO)的实现,是通过 Repository 作为 DAO 层接口来操作实体类所对应的数据表。Repository 模式是领域驱动设计中另一个经典的模式。当一个接口继承 JpaRepository 接口之后便自动具备了一系列常用的数据操作方法。因此,在抽象出业务流程中的实体(Entity)以后,在其实体类加上  \verb!@Entity! 的注解,便可以将实体类与 Repository 接口相关联,并提供了基础的增删改查,乃至分页等,实际业务中常见的基础数据库操作。这样的实现极大的省去了后端开发中的一部分基础代码编写,减轻了后端开发过程中的心智负担。